%%%%%%%%%%%%%%%%%%%%%%%%%%%%%%%%%%%%%%%%%
% Lachaise Assignment
% LaTeX Template
% Version 1.0 (26/6/2018)
%
% This template originates from:
% http://www.LaTeXTemplates.com
%
% Authors:
% Marion Lachaise & François Févotte
% Vel (vel@LaTeXTemplates.com)
%
% License:
% CC BY-NC-SA 3.0 (http://creativecommons.org/licenses/by-nc-sa/3.0/)
%
%%%%%%%%%%%%%%%%%%%%%%%%%%%%%%%%%%%%%%%%%

%----------------------------------------------------------------------------------------
%	PACKAGES AND OTHER DOCUMENT CONFIGURATIONS
%----------------------------------------------------------------------------------------

\documentclass{article}

\input{structure.tex} % Include the file specifying the document structure and custom commands

%----------------------------------------------------------------------------------------
%	ASSIGNMENT INFORMATION
%----------------------------------------------------------------------------------------

\title{COMP9417: Homework Set \#2} % Title of the assignment

\author{z5113817} % Author name and email address

\date{University of New South Wales --- \today} % University, school and/or department name(s) and a date

\newcommand\simplelrg{\hat{\beta}_{1} = \frac{\bar{XY} - \bar{X}\bar{Y}}{\bar{(X^2)} - (\bar{X})^2}}

\newcommand\sumlrg{\frac{1}{n}\sum_{i=1}^{n}(}
\newcommand\expandedlrg{\hat{\beta}_{1} = \frac{\sumlrg{}X_{i} - \bar{X})(Y_{i} - \bar{Y})}{\sumlrg{}X_{i} - \bar{X})^2}}

%----------------------------------------------------------------------------------------

\begin{document}

% \maketitle % Print the title

%----------------------------------------------------------------------------------------
%	Main Contents
%----------------------------------------------------------------------------------------

% All code for this homework set is available \href{https://github.com/william-coulter/COMP9417\_Homework\_2/tree/master}{here}.

% \newpage

\section*{Question 1}

\subsection*{a}

\includegraphics[scale=0.8]{NPBootstrap.png}

Screenshot of code here: \ref{code:q1a}.\\

i. C is a hyperparameter used for regularization to prevent overfitting of the Logistics Regression
model. This is inversely proportional to the penalty constant \(\lambda\) and penalises
models that have a lot of features and therefore the effect is that it reduces each feature importance.\\

ii. The effect that this has on the Bootstrap graph is that it increases the variance of 
each feature, therefore making the 90\% confidence intervals larger and more reliable since their largeness means they 
are more likely to include the true value.\\
At \(C = 0.1\), most of the features have their average at 0 with a very small confidence interval. Because I know 
how the data was generated and I know that the mean for each feature \emph{should} be 0, 
I know that this is a reliable estimate. However in the real world when I don't know 
how the data was generated, a model with \(C = 0.1\) would probably be overfit on the 
data and not contain a true mean.\\

\newpage

\subsection*{b}

\includegraphics[scale=0.8]{NPParameterisedBootstrap.png}

Screenshot of code here: \ref{code:q1b}.


\newpage
\section*{Appendix}

\subsection{q1a}
\label{code:q1a}
\includegraphics[scale=0.4]{q1a.png}

\subsection{q1b}
\label{code:q1b}
\includegraphics[scale=0.4]{q1b_code1.png}

\includegraphics[scale=0.3]{q1b_code2.png}

\end{document}
